\chapter{Jövőbeli munka}\label{chap:conclusion}

A kapott eredményeket még nem lehet egy az egyben a REHAROB kézmodulján alkalmazni. Következő 
lépésként általánosítani lehetne a modellt egy többszabadságfokú rendszerre, ami jobban modellezi 
a valódi kézmodult. A kapott diszkrét idejű stabilitástérképet alaposabban be lehetne járni, 
a motorparaméterek pontosabb meghatározása után. Különbőző időkésésekre össze lehetne hasonlítani 
a mért és a szimulált beállási időket. A stabilitás határán ki lehetne mérni a rezgési frekvenciát, 
illetve annak megváltozását a határon végighaladva. A beállási időn kívül más rendszerparamétereket 
is meg lehetne vizsgálni, mint a felfutási idő vagy a maximális túllövés. Kiterjeszthető a modell 
nemlineáris hatások bevonásával, ilyen lehetne a Coulomb-súrlódás vagy a szaturáció, esetleg a 
szabályozó jel diszkretizációja. Aztán a külső nyomaták kompenzációja ugyan bekerült a 
szabályozó modelljébe, de nem maradt idő a kompenzáció tesztelésére. Végül előfordulhat, 
hogy az impedancia modell paramétereit online kell változtatni, például ha "puhább" választ 
szertnénk kapni egy felülettel való kontakt során. Ennek az implementációját és a stabilitásra 
gyakorolt hatását is külön meg kell vizsgálni.
