\chapter{Stabilitásvizsgálat időkéséssel}\label{chap:time_delay_stability}

\section{Vizsgálati módszerek összehasonlítása}
A rúd differenciálegyenlete
\begin{align}
    \ddot\phi\left(t\right) - \frac{6g}{l}\phi\left(t\right) + 
    \frac{6D}{ml}\dot\phi\left(t-\tau\right) + \frac{6P}{ml}\phi\left(t-\tau\right) = 0
\end{align}
A differenciálegyenlet Laplace transzformáltja
\begin{align}
    s^2\phi\left(s\right) - s\phi_0 - \dot\phi_0 - 
    \frac{6g}{l}\phi\left(s\right) + 
    \frac{6D}{ml}\left(se^{-s\tau}\phi\left(s\right)-\phi_{-\tau}\right) + 
    \frac{6P}{ml}e^{-s\tau}\phi\left(s\right) = 0 
\end{align}
Kifejezve $\phi\left(s\right)$-t
\begin{align}
    \phi\left(s\right) = \frac{s\phi_0+\dot\phi_0+\frac{6D}{ml}\phi_{-\tau}}{s^2+\frac{6D}{ml}se^{-s\tau}+\frac{6P}{ml}e^{-s\tau}-\frac{6g}{l}}
\end{align}
A végérték frekvenciatartománybeli reprezentációban
\begin{align}
    \lim_{t \to \infty}\phi\left(t\right) = 
    \lim_{s \to 0} \phi\left(s\right) = 
    \frac{\dot\phi_0+\frac{6D}{ml}\phi_{-\tau}}{\frac{6P}{ml}-\frac{6g}{l}}
\end{align}
Az időkésést Taylor-sorral közelítve
\begin{align}
    \phi\left(t-\tau\right) = \phi\left(t-\right) - 
    \frac{1}{1!}\dot\phi\left(t\right)\tau + 
    \frac{1}{2!}\ddot\phi\left(t\right)\tau^2 - 
    \frac{1}{3!}\dddot\phi\left(t\right)\tau^3 + \ldots
\end{align}
különböző rendű közelítéssekkel

\section{Stabilitás folytonos időben}

\section{Stabilitás diszkrét időben}