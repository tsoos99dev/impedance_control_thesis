\chapter{Impedancia modell}\label{impedance_model}
Az eredményes ember-robot interakció érdekében a szabályozó előírása nem csupán 
az elérni kívánt pozíció vagy kifejtett nyomaték, hanem a mozgásállapot és a kifejtett
nyomaték közötti összefüggés. Ezt az összefüggést (linearitása végett) egy 
tömeg-rugó-csillapitás modell adja meg a továbbiakban, mely a következő alakban
írható fel: 
\begin{align}
    M_\RM e \ddot \theta + B_\RM e \dot \theta + K_\RM e (\theta - \theta_\RM r) = \tau_\RM e\,.
\end{align}
A modell három paraméterrel rendelkezik, $M_\RM e$ a rendszer előírt tehetetlensége, 
$B_\RM e$ a viszkózus csillapítása, $K_\RM e$ a rugóállandója. 
$\theta_\RM r$ és $\tau_\RM e$ az elérni kívánt pozíció és a rendszerre ható külső nyomaték. 