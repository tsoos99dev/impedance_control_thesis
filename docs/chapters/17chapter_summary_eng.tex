\chapter*{Summary}\label{chap:summary_eng}

\begin{otherlanguage}{USenglish}
In this paper, a simplified model of the hand module of the REHAROB rehabilitation robot is presented. 
Starting from a permanent magnet DC motor, through a hybrid position-torque controller, it was possible to 
enforce the motion prescribed by a second-order model on the system. By using a minimum-order observer, 
the controller minimizes the number of external sensors required. According to the condition formulated based on the settling time, 
the prescribed inertia of the model, the viscous damping coefficient, and the poles of the controller can be related. 
The prescribed inertia can be at most 2.5 times that of the physical system. Its minimum is determined by the expected 
settling time and the smallest pole that can be set for the controller.

I examined the stability of the model, augmented with time delay, in both continuous and discrete time. 
In both cases, a new relationship was found primarily between the viscous damping coefficient prescribed by 
the model and the spring constant. In the continuous case, it is possible to apply an arbitrarily large 
viscous damping coefficient with sufficiently small time delay. However, there exists a critical time delay 
above which the stability range is closed. In this case, a maximum is also imposed on the viscous damping coefficient. 
In discrete time, the stability range could only be determined numerically through simulations. For the parameters examined, 
the range was always closed. After deriving the state transition matrix of the discretized system, 
the relationship between the real and prescribed settling time can be determined by examining the eigenvalues of the resulting matrix. 
Taking this into account, the prescribable parameter combinations lie within a much smaller range compared to the entire stable area. 
Thus, considering the time delay greatly influences the overall system design process. As a result, motor selection and the design of 
the hardware implementing the controller can be better optimized. The system can become more cost-effective and safer, which is particularly 
important in applications requiring human-robot interaction.

\end{otherlanguage}
