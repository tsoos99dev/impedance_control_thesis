\chapter{Összegzés}\label{chap:summary}

A dolgozatban a REHAROB rehablitációs robot kézmoduljának egyszerűsített modellje került bemutatásra. Egy állandó 
gerjesztésű egyenáramú motorból kiindulva, egy hibrid pozíció-nyomaték szabályozón keresztül, sikerült egy másodrendű 
modell által előírt mozgást rákényszeríteni a rendszerre. A szabályozó egy minimumrendű megfigyelőt 
felhasználva minimalizálja a beépítendő külső szenzorok számát. A beállási idő alapján megalkotott feltétel 
szerint felírható a modell előírt tehetetlensége, viszkózus csillapítási együtthatója és a szabályozó 
pólusai közötti kapcsolat. Az előírt tehetetlenség legfeljebb a fizikai rendszer 2.5-szerese lehet. A minimumát az 
elvárt beállási idő és a szabályozónak beállítható legkisebb pólus határozza meg. 

Az időkéséssel kiegészített modell stabilitását folytonos és diszkrét időben is megvizsgáltam. Mindkét 
esetben elsősorban a modell által előírt viszkózus csillapítási együttható és a rúgóállandó közti 
kapcsolatra adódott új összefüggés. Folytonos esetben elegendően kis időkésésnél lehetséges, hogy tetszőlegesen 
nagy viszkózus csillapítási együttható alkalmazható. Azonban létezik egy kritikus időkésés, ami felett 
a stabilitási tartomány zárt. Ekkor a viszkózus csillapítási együtthatóra is adódik egy maximum. Diszkrét időben csak numerikusan, szimulációkkal sikerült a stabilitási tartományt
meghatározni. A vizsgált paraméterekkel a tartomány mindig zárt volt. A diszkretizált rendszer állapot átmeneti mátrixának 
levezetése után a kapott mátrix sajátértékeinek
vizsgálatával meghatározható a valós és az előírt beállási idő közötti kapcsolat. Ezt is figyelembe véve az 
előírható paraméterkombinációk jóval kisebb tartományban helyezkednek el a teljes stabil területhez képest. Tehát 
az időkésés figyelembe vétele nagyban befolyásolja a teljes rendszertervezési folyamatot. Végeredményben jobban 
optimalizálható a motorválasztás és a szabályozót implementáló hardver megtervezése. Költséghatékonyabb és biztonságosabb 
lehet a rendszer, ami egy ember-robot interakciót igénylő alkalmazásban különösen fontos. 

