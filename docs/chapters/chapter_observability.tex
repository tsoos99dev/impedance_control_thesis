\chapter{Megfigyelhetőség}\label{chap:observability}
A felhasznált szenzorok minimalizálása érdekében egyetlen kimenet mérése a cél, így a kimenetek közül egyedül a 
szögelfordulás áll elő a továbbiakban közvetlen mérésből. A szabályozó teljes állapotvisszacsatolásra épül, 
így a kimenet mérésével minden állapot megfigyelhető kell legyen. 
Az~\eqref{eq:state_space_generic} és~\eqref{eq:state_space} egyenletek alapján a kimeneti megfigyelhetőség feltétele, hogy
\begin{align}
    \left[\begin{array}{c}
        \BF{C} \\ \hline
        \BF{CA} \\ \hline
        \BF C \BF A^2
    \end{array}\right]
\end{align}
mátrix legyen maximális rangú. A feltételben szereplő mátrixot a motorparaméterekkel kifejezve~\eqref{eq:state_space} alapján:
\begin{align}
    \begin{bmatrix}
        1 & 0 & 0 \\
        0 & 1 & 0 \\
        0 & -\frac{B_\RM m}{J} & \frac{K_\RM m}{J}
    \end{bmatrix},
\end{align}
mely redukált lépcsős alakban
\begin{align}
    \begin{bmatrix}
        1 & 0 & 0 \\
        0 & 1 & 0 \\
        0 & 0 & 1
    \end{bmatrix},
\end{align}
ami valóban maximális rangú, tehát a rendszer minden állapota megfigyelhető a szögelfordulás méréséből.
