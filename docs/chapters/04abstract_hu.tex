\cleardoublepage{}
\phantomsection{}
\pdfbookmark[0]{Kivonat}{kivonat}
\chapter*{Kivonat}

A REHAROB egy egészségügyi robot, mely többek között a sztrók okozta mozgásszervi 
zavar rehabilitációjában alkalmazható. 
%
\replace{
	Az eddigi tesztek során a robot többször került közel instabil állapotba. A stabilitásvesztés könnyen a pácienst is veszélyeztető szituációt idézhet elő.%
}{
	Minden digitálisan szabályozott rendszerhez hasonlóan a REHAROB rendszerben is felmerülhetnek a digitális mintavételezésből adódó jelenségek mint például a digitális mintavételezési időkésésből eredő rezgések melyek a szabályozó stabilitásvesztéséhez vezethet. 
	%
	Egy esetleges stabilitásvesztés pedig könnyen a pácienst is veszélyeztető szituációt idézhet elő.%
}
%


Ebben a dolgozatban a robot kézmoduljának egy egyszerűsített egy dimenziós modelljére készült egy admittancia szabályozó. 
%
A szabályozó pozíció és nyomaték bemenettel 
dolgozik, és egy virtuális másodrendű tömeg-rugó-csillapítás mozgását képes emulálni. 
%
A dolgozat az időkésés stabilitásra gyakorolt hatását tárgyalja mind
folytonos, mind pedig diszkrét időben. 