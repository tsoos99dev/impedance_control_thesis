\chapter{Irányíthatóság}\label{chap:controllability}
A szabályozó akkor tudja követni a számára előírt impedancia modellt, 
ha megfelelő bemeneti feszültség alkalmazásával eljutttatható az előírt állapotba. 
Az~\eqref{eq:state_space_generic}-os állapottér modell alapján a kimeneti irányíthatóság feltétele, hogy
\begin{align}\label{eq:observability_generic}
    \left[\begin{array}{c|c|c|c}
        \bm{CB} & \bm{CAB} & \bm C \bm A^2 \bm B & \bm D
    \end{array}\right]
\end{align}
legyen maximális rangú. Ez csak egy szükséges, de nem elégséges feltétele az impedancia modell alkalmazhatóságának.
Felhasználva~\eqref{eq:state_space} paramétereit a~\eqref{eq:observability_generic}-es feltételben szereplő mátrix
\begin{align}
    \begin{bmatrix}
        0 & 0 & \frac{K_m}{JL} & 0
    \end{bmatrix},
\end{align}
alakba írható át. Továbbá ez a mátrix redukált lépcsős alakban
\begin{align}
    \begin{bmatrix}
        0 & 0 & 1 & 0
    \end{bmatrix},
\end{align}
mely mátrix rangja megegyezik sorainak számával, így az irányíthatóság feltétele teljesül.