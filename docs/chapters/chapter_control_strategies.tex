\chapter{Control strategies} \label{chap:control}


The control of Furuta pendulum is well-researched topic since its introduction. The two main subjects are the stabilization around the unstable equilibrium point and the swing up control \cite{furuta1991swing,furuta1992swing}. 


The stabilization around the unstable equilibrium seems to be a minor task, because similar linearized equations can be obtained that of the planar pendulum's by using appropriate assumptions. Because this similarity, the later presented results are valid for both types of pendulums.

However, it is worth keep in mind  that the original nonlinear equations are much more complicated in the case of Furuta pendulum. Additional nonlinear terms represent the special coupling between the coordinates (see Section \ref{sec:eq_of_mot_full}). This can cause extra performance penalty on the controller.


Here we investigate both ideal analog and real digital control strategies. In both cases we considered the effect of the electric DC motor as a real actuator.





%%%%%%%%%%%%%%%%%%%%%%%%%%%%%%%%%%%%%%%%%%%%%%%%%%%%%%%%%%%%%%%%%%%%%%%%%%%%%%%%
%%%%%%%%%%%%%%%%%%%%%%%%%%%%%%%%%%%%%%%%%%%%%%%%%%%%%%%%%%%%%%%%%%%%%%%%%%%%%%%%
%%%%%%%%%%%%%%%%%%%%%%%%%%%%%%%%%%%%%%%%%%%%%%%%%%%%%%%%%%%%%%%%%%%%%%%%%%%%%%%%
%%%%%%%%%%%%%%%%%%%%%%%%%%%%%%%%%%%%%%%%%%%%%%%%%%%%%%%%%%%%%%%%%%%%%%%%%%%%%%%%
%%%%%%%%%%%%%%%%%%%%%%%%%%%%%%%%%%%%%%%%%%%%%%%%%%%%%%%%%%%%%%%%%%%%%%%%%%%%%%%%
%%%%%%%%%%%%%%%%%%%%%%%%%%%%%%%%%%%%%%%%%%%%%%%%%%%%%%%%%%%%%%%%%%%%%%%%%%%%%%%%
%\clearpage
\section{Governing equations of electric motor}\label{Sec:electric_motor}

As the actuator is a real electric motor, we may have to consider the inductance of the armature, the back electromagnetic force (later just back-EMF) and the viscous friction on the motor bearings. The governing equations of the motor are

\begin{align}
%	\begin{subequations}
		L_a \dfrac{\diff i_a}{\diff t} + R_a i_a + K_v \omega &= \Uin\,, \\[2ex]
		J \dfrac{\diff \omega}{\diff t} + b_m \omega - K_v i_a &= 0\,, 
%	\end{subequations}
\end{align}

\noindent where  $L_a$ and $R_a$ are the armature inductance and resistance respectively, $i_a$ is the armature current, $\Uin$ is the input voltage of the motor, $J$ is the total mass moment of inertia of the motor and the load, $b_m$ is the damping coefficient at the motor bearings, $K_v$ is the back-EMF or torque constant. The angular velocity of the motor is denoted by $\omega$; considering totally rigid connection between the motor and the arm of pendulum, this is equal to the angular velocity of the arm 
\begin{align}
	\omega = \dot \varphi\,.
\end{align}

The motor characteristics already takes into account the gear ratio in  the gearbox.
The simple but widely used expression of the output motor torque $M$ can be obtained by neglecting the viscous damping $b_m=0$ and the armature inductance $L_a = 0$; and combining the Newton--Euler equations ($J \ddot \varphi = M$) with the previous equations, this yields
\begin{align}
	M = \dfrac{K_v}{R_a}\Uin - \dfrac{K_{v}^{2}}{R_a} \dot \varphi = N \Uin - K \dot \varphi\,. \label{eq:electic_motor}
\end{align}



%%%%%%%%%%%%%%%%%%%%%%%%%%%%%%%%%%%%%%%%%%%%%%%%%%%%%%%%%%%%%%%%%%%%%%%%%%%%%%%%
%%%%%%%%%%%%%%%%%%%%%%%%%%%%%%%%%%%%%%%%%%%%%%%%%%%%%%%%%%%%%%%%%%%%%%%%%%%%%%%%
%%%%%%%%%%%%%%%%%%%%%%%%%%%%%%%%%%%%%%%%%%%%%%%%%%%%%%%%%%%%%%%%%%%%%%%%%%%%%%%%
%%%%%%%%%%%%%%%%%%%%%%%%%%%%%%%%%%%%%%%%%%%%%%%%%%%%%%%%%%%%%%%%%%%%%%%%%%%%%%%%
%%%%%%%%%%%%%%%%%%%%%%%%%%%%%%%%%%%%%%%%%%%%%%%%%%%%%%%%%%%%%%%%%%%%%%%%%%%%%%%%
%%%%%%%%%%%%%%%%%%%%%%%%%%%%%%%%%%%%%%%%%%%%%%%%%%%%%%%%%%%%%%%%%%%%%%%%%%%%%%%%
\section{Analog control}


%%%%%%%%%%%%%%%%%%%%%%%%%%%%%%%%%%%%%%%%%%%%%%%%%%%%%%%%%%%%%%%%%%%%%%%%%%%%%%%%
%%%%%%%%%%%%%%%%%%%%%%%%%%%%%%%%%%%%%%%%%%%%%%%%%%%%%%%%%%%%%%%%%%%%%%%%%%%%%%%%
%%%%%%%%%%%%%%%%%%%%%%%%%%%%%%%%%%%%%%%%%%%%%%%%%%%%%%%%%%%%%%%%%%%%%%%%%%%%%%%%
%\clearpage
\subsection{Back-EMF effect on 1 DoF position control stability}

The effect of back-EMF is first examined in the simplest possible case, a one dimensional linear position control case. This is presented in Figure \ref{fig:1dof_pos_cont}, where $m$ denotes the mass, $x$ is the position, $\Uin$ is the input voltage of the motor, $M$ is the output torque, $j = l/2\pi$ and $l$ is lead the ball screw lead. The friction is neglected $\mu\approx 0$.


\begin{figure}[h]
	\centering
	%	\includegraphics[width=0.99\linewidth]{FURUTA_ALL}
	\def\svgwidth{10cm}
	\input{./figures/one_dof_pos_control_jav.pdf_tex}
	\caption{Basic model for examining the back-EMF effect for 1 DoF linear position control.}
	\label{fig:1dof_pos_cont}
\end{figure}

\noindent
The connection between the force acting on the mass and the motor torque, the motor position $\varphi$ and mass position $x$, and the angular velocity of the motor $\omega$ and velocity $\dot x$ of the mass  are the followings:
\begin{align}
	Q &= \dfrac{1}{j}M\,, \\
	x &= j \varphi\,, \\
	\dot x &= j \dot \varphi = j \omega\,.
\end{align}

\noindent
The equaton of motion in this case is
\begin{align}
 m\ddot x = Q\,.
 \end{align}
After the substitution of the electric DC motor equation (\ref{eq:electic_motor}) and the above mentioned connections into the equation of motion, we obtain
 \begin{align}
 	m\ddot x = \dfrac{N}{j}\Uin - \dfrac{K}{j^2} \dot x\,.
 \end{align}


\noindent 
For position control, the usual traditional controller type is the PD controller. Assuming that the desired position is at $x_\mathrm{d}=0$ and using a PD controller the input voltage of the motor is
\begin{align}
	\Uin = -Px -D\dot x\,, \label{eq:PD_controller}
\end{align} 
where $P$ and $D$ are the parameters of the controller. After using this for the equation of motion and a rearrangement, we get
\begin{align}
	m\ddot x + \dfrac{K + jD}{j^2}\dot x + \dfrac{NP}{j} x = 0\,.
\end{align} 

\noindent
This is a homogeneous linear differential equation with constant coefficients, we can use the exponential trial solution:
\begin{align}
	x(t) 	  	&= C e^{\lambda t} \,, 			\\[1ex]
	\dot x(t) 	&= \lambda C  e^{\lambda t}	\,,	\\[1ex]
	\ddot x(t) 	&= \lambda^2 C  e^{\lambda t}\,.	
\end{align}

\noindent
By putting these back to the differential equation, we get the characteristic equation. We search for a non trivial solution $C \neq 0$, so the characteristic polynomial has to equal to zero: 

\begin{align}
	 \underbrace{\left( m \lambda^2  +\dfrac{K+Dj}{j^2} \lambda + \dfrac{NP}{j} \right)}_{\displaystyle =0} Ce^{\lambda t} = 0\,.
\end{align}

\noindent
The roots of the characteristic polynomial $\lambda_{1,2} \in \mathbb C$ are called characteristic exponents. The condition for exponential stability of equilibrium point $x = 0$ is that the characteristic roots have negative real part
\begin{align}
	\mathrm{Re}\, \lambda_{1,2} < 0\,.
\end{align}

\noindent This can be checked by the Routh-Hurwitz criterion. This states that all roots of an $n^\mathrm{th} $ degree polynomial $a_n\lambda^n +a_{n-1}\lambda^{n-1} +..+a_0=0$ are placed in the negative real half-plane if and only if all coefficients and special products of them called Hurwitz determinants are also positive: 
\begin{align}
	\mathrm{Re}\, \lambda_{i} < 0\, i=1,2,..,n < 0 \iff \begin{cases}
		a_j > 0,\ j = 0,1,2,..,n \\
		\det \mathbf{H}_k >0, \ k = 1,2,..,n 
	\end{cases}
\end{align}

%\noindent A more detailed summary about this Routh-Hurwitz stability criterion and the Hurwitz matrices can be found in Appendix \ref{app:routh_hurwitz}. 
\noindent We can conclude that the system is stable if the following inequalities are fulfilled:
\begin{alignat}{3}
	a_2&=m>0 \,,				\\
	a_1&=\dfrac{K+Dj}{j^2}>0 &&\rightarrow D&&>-\dfrac{K}{j}\,, \label{eq:motor_D_condition} \\
	a_0&=\dfrac{NP}{j}>0	 &&\rightarrow P&&>0 \,.
\end{alignat}
	
\noindent 
The back-EMF constant is positive $K>0$, so we can conclude that it helps the stability, because it adds additional damping to the system. This is why they like to use the DC motor for simple position control, the back-EMF effect helps to stabilize the system in this case.

%%%%%%%%%%%%%%%%%%%%%%%%%%%%%%%%%%%%%%%%%%%%%%%%%%%%%%%%%%%%%%%%%%%%%%%%%%%%%%%%
%%%%%%%%%%%%%%%%%%%%%%%%%%%%%%%%%%%%%%%%%%%%%%%%%%%%%%%%%%%%%%%%%%%%%%%%%%%%%%%%
%%%%%%%%%%%%%%%%%%%%%%%%%%%%%%%%%%%%%%%%%%%%%%%%%%%%%%%%%%%%%%%%%%%%%%%%%%%%%%%%

%\clearpage
\subsection{Stabilizing Furuta pendulum with PD controller}

The natural idea for stabilizing the upward position of the Furuta pendulum as a position control  task is to use PD controller as the simplest choice. 
The linearized equation of motion with neglected damping terms is
\begin{align}
		\left(\dfrac{J_a J_p}{m_2 r l}- m_2 r l\right) \ddot \theta + \left(-m_2 r l - \dfrac{g}{r}\right) \theta = M\,. 
		\label{eq:lienarized_essential_equation_wo_damping}
\end{align}

\noindent Using the electric motor equation (\ref{eq:electic_motor}) for the torque $M$ and the PD controller equation (\ref{eq:PD_controller}) for the motor input voltage $\Uin$ with variable $\theta$ instead of $x$ the new equation is
%
\begin{align}
	\left(\dfrac{J_a J_p}{m_2 r l}- m_2 r l\right) \ddot \theta + \left(-m_2 r l - \dfrac{g}{r}\right) \theta &=  -NP\theta -ND\dot{\theta} -K\dot{\varphi}\,,
\end{align}

\noindent
where we can observe that the dependence on the cyclic angular velocity $\dot \varphi$ is present again due to the back-EMF effect. To eliminate this, we have to derivate the equation 
\begin{align}
 	\left(\dfrac{J_a J_p}{m_2 r l}- m_2 r l\right) \dddot \theta + \left(-m_2 r l - \dfrac{g}{r}\right) \dot \theta &= -NP\dot \theta -ND\ddot{\theta} -K\ddot{\varphi}\,,
\end{align}

\noindent
and to use the equation (\ref{eq:ddot_varphi_linearized_expressed}) to replace the cyclic angular acceleration term $\ddot \varphi$. After a rearrangement, we get a 3\textsuperscript{rd} order linear homogeneous differential equation
\begin{align}
	\left(\dfrac{J_a J_p}{m_2 r l}- m_2 r l\right) \dddot \theta 
	+
	\left(\dfrac{KJ_p}{m_2 r l} + ND\right) \ddot\theta
	+
	 \left(NP - \dfrac{J_a g}{r}\right) \dot \theta
	+
	\left(-\dfrac{Kg}{r}\right) \theta = 0\,. \label{eq:PD_diffeq}
\end{align}

\noindent
Using the exponential trial solution $\theta(t) = Ce^{\lambda t} $, the characteristic polynomial becomes
\begin{align}
		\left(\dfrac{J_a J_p}{m_2 r l}- m_2 r l\right) \lambda^3
		+
		\left(\dfrac{KJ_p}{m_2 r l} + ND\right) \lambda^2
		+
		\left(NP - \dfrac{J_a g}{r}\right) \lambda
		+
		\left(-\dfrac{Kg}{r}\right) = 0\,, \label{eq:PD_charpoly}
\end{align}

\noindent 
for which the Routh-Hurwitz criterion is applicable to determine the stability of the upward position of the pendulum. In this case, because the polynomial is 3\textsuperscript{rd} order, we have to take into account a new condition for the 2\textsuperscript{nd} Hurwitz determinant:
\begin{subequations}
	\begin{alignat}{3}
		a_3&=\dfrac{J_a J_p}{m_2 r l}- m_2 r l>0\,, 			\label{eq:PD_analog_a3}	\\
		a_2&=\dfrac{KJ_p}{m_2 r l} + ND>0 &&\quad\rightarrow\quad D&&>-\dfrac{K J_p}{N m_2 r l}\,, \\
		a_1&=NP - \dfrac{J_a g}{r}>0 &&\quad\rightarrow\quad P&&>\dfrac{J_a g}{N r} \,, \\
		a_0&=-\dfrac{Kg}{r}>0\,, \\
		\det(\mathbf H_2) &= a_2a_1-a_3a_1 > 0\,.
	\end{alignat}
\end{subequations}


\noindent
The first condition can be verified by substituting back the introduced $J_p$ and $J_a$ values from equation (\ref{eq:Jp}) and (\ref{eq:Ja}), respectively: 
\begin{align}
	a_3&=\dfrac{J_a J_p}{m_2 r l}- m_2 r l = \dfrac{(m_1 l_1^2 + J_{1z}  + m_2 r^2) (m_2 l^2 + J_{2y_3}) }{m_2 r l} -m_2 r l  \nonumber\\
	&= 	\dfrac{ m_1 m_2 l_1^2l^2 
			+ m_1 l_1 J_{2y_3} 
			+m_2 l^2 J_{1z}
			+m_2 r^2 J_{2y_3}
			+J_{1z} J_{2y_3} 
			}{m_2 r l}\,. \label{eq:a_3_is_positive}
\end{align}

\noindent
Because every term is positive in this expanded form of coefficient $a_3$, the whole expression is positive as well. There are no controller parameter in coefficient $a_3$, so the (\ref{eq:PD_analog_a3}) condition is satisfied automatically. 

%\noindent
The conditions for the coefficients $a_2$ and $a_1$ give lower boundaries for  parameters $P$ and $D$. The back-EMF  effect enlarge the stability range for the  parameter $D$, this is similar to the previously shown 1 DoF position control example. However, the stable  parameter $P$ range has shrunk, because the proportional term has to compensate the gravity.

%noidennt
The last coefficient $a_0$ is interesting, every variable is positive, so the $a_0>0$ inequality is a contradiction. The back-EMF effect appears together with the gravity in this coefficient, but there are no controller parameters inside for compensation.

%\noident
At this point we can conclude that the inverted pendulum and the Furuta pendulum cannot be stabilized with a traditional PD controller around the upward position, if the  back-EMF effect of the electric DC motor is considered.
We can think that the real system has damping and it can help, but the back-EMF effect is similar to the viscous friction, so this would also destabilize our system. For getting stability charts we have to look for a more complicated control strategy, like PID or \PPDD{} controller.

Note that, we have not finished the stability analysis, the last condition for the 2\textsuperscript{nd} Hurwitz determinant should also be checked (it is always true), it just lost its importance due to the previous contradiction.

There is one more thing we can hope for, the real controllers cause discrete time delay in the system and there are existing physical examples when this can stabilize the system, this is discussed later.





%%%%%%%%%%%%%%%%%%%%%%%%%%%%%%%%%%%%%%%%%%%%%%%%%%%%%%%%%%%%%%%%%%%%%%%%%%%%%%%%
%%%%%%%%%%%%%%%%%%%%%%%%%%%%%%%%%%%%%%%%%%%%%%%%%%%%%%%%%%%%%%%%%%%%%%%%%%%%%%%%
%%%%%%%%%%%%%%%%%%%%%%%%%%%%%%%%%%%%%%%%%%%%%%%%%%%%%%%%%%%%%%%%%%%%%%%%%%%%%%%%
%\clearpage
\subsection{\texorpdfstring{P\textsubscript{1}D\textsubscript{1}D\textsubscript{2} controller}{P\_1D\_1D\_2 controller}}

One possibility to improve the control law is to use the cyclic coordinate $\varphi$ and velocity $\dot{\varphi}$ also with new proportional and differential terms. This way we get the \PPDD{} controller, where subscript 1 and 2 stand for the essential and cyclic coordinate, respectively. The general control law in this case is
\begin{align}
	\Uin(\theta, \dot\theta, \varphi, \dot{\varphi}) = -P_1\theta -D_1\dot\theta + P_2\varphi +D_2\dot\varphi\,. \label{eq:PPDD_controller}
\end{align}

%\noindent
This has 4 control parameters $P_1, D_1, P_2, D_2$, which later would result in a 4 dimensional stability chart. It is much more complicated to  say something qualitatively about them, than about the simple 2 dimensional charts. Moreover, this would also increase the order of the differential equation to 4\textsuperscript{th} order, and the degree of the characteristic polynomial to 4\textsuperscript{th} degree as well. This would lead to a 4 by 4 Hurwitz matrix, which gives more condition to check. 

The motor equation (\ref{eq:electic_motor}) does not depend on the cyclic coordinate, but the cyclic velocity, thus we neglect the controller term $P_2$. This logic results in the \PDD{} controller
\begin{align}
		\Uin(\theta, \dot\theta, \dot{\varphi}) = -P_1\theta -D_1\dot\theta + D_2\dot\varphi\,. \label{eq:PDD_controller}
\end{align}

This simplification results in a 3 dimensional chart, which is still complicated, but the differential equation remains 3\textsuperscript{rd} order, the characteristic polynomial remains 3\textsuperscript{rd} degree as before.


We have to note that it can be important to keep the proportional term of the cyclic position $P_2$ in real experiments, for example in case if someone wants to stabilize the inverted pendulum in a specific $\varphi$ position, not just somewhere. It can be also useful in the case of wheeled pendulums to prevent falling off the table. 
%
%\textcolor{red}{[ESETLEG HIVATKOZÁS A RÉGI CIKKRE] , they just forgot to mention this cause.}
%





To start the stability analysis we begin with the linearized equation of motion
\begin{align*}
\left(\dfrac{J_a J_p}{m_2 r l}- m_2 r l\right) \ddot \theta + \left(-m_2 r l - \dfrac{g}{r}\right) \theta = M\,. \tag{\ref{eq:lienarized_essential_equation_wo_damping} revisited}
\end{align*}

\noindent
By substituting the motor equation (\ref{eq:electic_motor}) and the new  \PDD{} conrtol law (\ref{eq:PDD_controller}) into this, we get
\begin{align}
\left(\dfrac{J_a J_p}{m_2 r l}- m_2 r l\right) \ddot \theta + \left(-m_2 r l - \dfrac{g}{r}\right) \theta &=  -NP_1\theta -ND_1\dot{\theta} + ND_2 \dot{\varphi} -K\dot{\varphi}\,.
\end{align}

\noindent
We have to eliminate the cyclic velocity $\dot{\varphi}$ again by derivation and using the equation (\ref{eq:ddot_varphi_linearized_expressed}) to obtain 
\begin{align}
\left(\dfrac{J_a J_p}{m_2 r l}- m_2 r l\right) \dddot \theta 
&+
\left(\dfrac{KJ_p}{m_2 r l} + ND_1 - \dfrac{D_2 N J_p}{m_2 r l}\right) \ddot\theta
\nonumber \\
&+
\left(NP_1 - \dfrac{J_a g}{r}\right) \dot \theta
+
\left(\dfrac{D_2 N g}{r}-\dfrac{Kg}{r}\right) \theta = 0\,. \label{eq:PDD_diffeq}
\end{align}

\noindent
Using the exponential trial solution $\theta(t) = Ce^{\lambda t} $ the characteristic polynomial becomes
\begin{align}
\left(\dfrac{J_a J_p}{m_2 r l}- m_2 r l\right) \lambda^3 
&+
\left(\dfrac{KJ_p}{m_2 r l} + ND_1 - \dfrac{D_2 N J_p}{m_2 r l}\right) \lambda^2
\nonumber \\
&+
\left(NP_1 - \dfrac{J_a g}{r}\right) \lambda
+
\left(\dfrac{D_2 N g}{r}-\dfrac{Kg}{r}\right) = 0\,, \label{eq:PDD_charpoly}
\end{align}

\noindent 
for which the Routh-Hurwitz criterion is applicable to determine the stability of the pendulum's upward position. In this case, because the polynomial is 3\textsuperscript{rd} order, we have to take into account the condition for the 2\textsuperscript{nd} Hurwitz determinant:
\begin{align}
a_3&=\dfrac{J_a J_p}{m_2 r l}- m_2 r l >0\,, 	\label{eq:PDD_analog_a3}	
\\
%
a_2&=\dfrac{KJ_p}{m_2 r l} + ND_1 - \dfrac{D_2 N J_p}{m_2 r l} >0\,,
\\
%
a_1&=NP_1 - \dfrac{J_a g}{r} >0\,, 				\label{eq:motor_DD_condition} \\
a_0&=\dfrac{D_2 N g}{r}-\dfrac{Kg}{r} >0 \,,						\\
\det(\mathbf H_2) &= a_2a_1-a_3a_1 > 0\,.
\end{align}

\noindent
The $a_3>0$ condition is satisfied, it is explained before and proved by equation (\ref{eq:a_3_is_positive}).
The $a_2>0$ condition includes both differential control parameters $D_1$ and $D_2$, the stable parameter range for one can be expressed as a function of the other. The $D_1$ term has a lower boundary, the $D_2$ term has an upper boundary:
\begin{align}
	D_1(D_2) &> \dfrac{J_p}{m_2 r l}D_2  - \dfrac{K J_p}{N m_2 r l}\,,\\
	D_2(D_1) &< \dfrac{m_2 r l}{ J_p} D_1  + \dfrac{K}{N}\,.
\end{align}

\noindent
The $a_1>0$ condition gives a lower boundary for the proportional term
\begin{align}
	P_1 > \dfrac{J_a g}{N r}\,.
\end{align} 

\noindent
The $a_0>0$ condition gives a lower boundary for the $D_2$ term, which means it has both lower and upper boundary
\begin{align}
	D_2 > \dfrac{K}{N}\,.
\end{align}


\begin{figure}[h]
	\centering
	\includegraphics[width=0.8\linewidth]{figures/PDD2_D2_all_SMALL.png}
	\caption{Stability chart for \PDD{} controller, the $D_2$ parameters are shown on the top of contour lines.}
	\label{fig:PDD2_D2_all}
\end{figure}

\noindent
The last ($\det \mathbf H_2>0$) condition depends on all the three controller parameters and gives a nonlinear expression in which $P_1 D_1$ and $P_1  D_2$ products appear:
\begin{align}
\left(\dfrac{KJ_p}{m_2 r l} + ND_1 - \dfrac{D_2 N J_p}{m_2 r l} \right)
\left(NP_1 - \dfrac{J_a g}{r} \right) \phantom{===}\nonumber\\
-
\left(\dfrac{J_a J_p}{m_2 r l}- m_2 r l \right)
\left(\dfrac{D_2 N g}{r}-\dfrac{Kg}{r} \right) >0 \,.
\end{align}



The stabilization of Furuta pendulum around the upward position considering the back-EMF effect becomes possible with the \PDD{} controller. An example stability chart is shown in Figure \ref{fig:PDD2_D2_all}.





%%%%%%%%%%%%%%%%%%%%%%%%%%%%%%%%%%%%%%%%%%%%%%%%%%%%%%%%%%%%%%%%%%%%%%%%%%%%%%%%
%%%%%%%%%%%%%%%%%%%%%%%%%%%%%%%%%%%%%%%%%%%%%%%%%%%%%%%%%%%%%%%%%%%%%%%%%%%%%%%%
%%%%%%%%%%%%%%%%%%%%%%%%%%%%%%%%%%%%%%%%%%%%%%%%%%%%%%%%%%%%%%%%%%%%%%%%%%%%%%%%

%\begin{comment}
%\clearpage
\subsection{PID controller}
The other method to improve the initial PD controller is to extend it with an integral term to get the PID controller. The integral term accumulates the position error. This control law uses only the essential coordinate and velocity so we can get rid of the measurement of the cyclic variables.
The control law for the PID controller is
\begin{align}
	\Uin(\theta, \dot\theta) = -P\theta -D\dot\theta - I \int_{0}^{t} \theta \diff \tau, \label{eq:PID_controller}
\end{align}

\noindent
which has three control parameters; proportional $P$, differential $D$, integral $I$. The stability chart is 3 dimensional and we do not have possibility to simplify the control law further.

Recall the linearized equation of motion for starting the stability analysis
\begin{align*}
	\left(\dfrac{J_a J_p}{m_2 r l}- m_2 r l\right) \ddot \theta + \left(-m_2 r l - \dfrac{g}{r}\right) \theta = M. \tag{\ref{eq:lienarized_essential_equation_wo_damping} revisited}
\end{align*}


\noindent
By substituting back the motor equation (\ref{eq:electic_motor}) and the PID control law (\ref{eq:PID_controller}) we get
\begin{align}
	\left(\dfrac{J_a J_p}{m_2 r l}- m_2 r l\right) \ddot \theta + \left(-m_2 r l - \dfrac{g}{r}\right) \theta &=  -NP\theta -ND\dot{\theta} - NI \int_{0}^{t} \theta \diff \tau -K\dot{\varphi}.
\end{align}


\noindent
We have to eliminate the cyclic velocity $\dot{\varphi}$ again by derivation and using the equation (\ref{eq:ddot_varphi_linearized_expressed}) to get an uncoupled differential equation for $\theta$ and its derivatives 
\begin{align}
	\left(\dfrac{J_a J_p}{m_2 r l}- m_2 r l\right) \dddot \theta 
	&+
	\left( ND + \dfrac{KJ_p}{m_2 r l}\right) \ddot\theta
	\nonumber \\
	&+
	\left(NP - \dfrac{J_a g}{r}\right) \dot \theta
	+
	\left(NI-\dfrac{Kg}{r}\right) \theta = 0. 
\end{align}

\noindent
Note that the new integral term $I$ appears only in one coefficient $a_0$, not like in the case of previous \PDD{} controller where the new $D_2$ term appears in two coefficient $a_2$ and $a_0$.

%\noindent
Using the exponential trial solution $\theta(t) = Ce^{\lambda t} $ the characteristic polynomial becomes
\begin{align}
	\left(\dfrac{J_a J_p}{m_2 r l}- m_2 r l\right) \lambda^3 
	&+
	\left( ND + \dfrac{KJ_p}{m_2 r l}\right) \lambda^2
	\nonumber \\
	&+
	\left(NP - \dfrac{J_a g}{r}\right) \lambda
	+
	\left( NI-\dfrac{Kg}{r}\right) = 0, \label{eq:PID_charpoly}
\end{align}

\noindent 
for which the Routh-Hurwitz criterion is applicable to determine the stability of the pendulum's upward position. In this case  the characteristic polynomial is 3\textsuperscript{rd} order so we have to take into account the condition for the 2\textsuperscript{nd} Hurwitz determinant similarly to the previous case.

\begin{alignat}{3}
	a_3&=\dfrac{J_a J_p}{m_2 r l}- m_2 r l &&>0 	\label{eq:PID_analog_a3}	
	\\
	%
	a_2&=ND +\dfrac{KJ_p}{m_2 r l} &&>0 \quad \rightarrow \quad D &&> -\dfrac{KJ_p}{N m_2 r l} \\
	%
	a_1&=NP - \dfrac{J_a g}{r} &&>0      \quad \rightarrow \quad P &&>	\dfrac{J_a g}{N r} \\
	a_0&=NI - \dfrac{Kg}{r} &&>0 	     \quad \rightarrow \quad I &&>	\dfrac{Kg}{N r}\\
	\det(\mathbf H_2) &= a_2a_1-a_3a_1 &&> 0
\end{alignat}

\begin{figure}[h]
	\centering
	\includegraphics[width=0.8\linewidth]{figures/PID_I_all_with_units_SMALL.png}
	\caption{Stability chart for PID controller, the $I$ parameters are shown on the top of contour lines. }
	\label{fig:PID_I_200}
\end{figure}

\noindent
The $a_3>0$ condition is satisfied, it is explained before and proved by equation (\ref{eq:a_3_is_positive}), the $a_i > 0\quad i\in [0,1,2]$ conditions give simple lower boundaries for $I,P,D$, respectively. The 2\textsuperscript{nd} Hurwitz determinant gives a nonlinear expression, the $PD$ product term appears inside.
\begin{align}
	-N^2 P D - \dfrac{NKJ_p}{m_2 r l} P + \dfrac{J_a g}{r}D   + Kg m_2 l < N\left(\dfrac{J_a J_p}{m_2 r l}- m_2 r l\right)I
\end{align}
\noindent
This gives an upper boundary for the integral term $I$ as a hyperboloid paraboloid surface over the $(P,D)$ plane.










%\textcolor{red}{remark(s) : \\
%	Columb súrlódás és Integráló tag bizergő beállás
%}
%\end{comment}
%%%%%%%%%%%%%%%%%%%%%%%%%%%%%%%%%%%%%%%%%%%%%%%%%%%%%%%%%%%%%%%%%%%%%%%%%%%%%%%%
%%%%%%%%%%%%%%%%%%%%%%%%%%%%%%%%%%%%%%%%%%%%%%%%%%%%%%%%%%%%%%%%%%%%%%%%%%%%%%%%
%%%%%%%%%%%%%%%%%%%%%%%%%%%%%%%%%%%%%%%%%%%%%%%%%%%%%%%%%%%%%%%%%%%%%%%%%%%%%%%%
%%%%%%%%%%%%%%%%%%%%%%%%%%%%%%%%%%%%%%%%%%%%%%%%%%%%%%%%%%%%%%%%%%%%%%%%%%%%%%%%
%%%%%%%%%%%%%%%%%%%%%%%%%%%%%%%%%%%%%%%%%%%%%%%%%%%%%%%%%%%%%%%%%%%%%%%%%%%%%%%%
%%%%%%%%%%%%%%%%%%%%%%%%%%%%%%%%%%%%%%%%%%%%%%%%%%%%%%%%%%%%%%%%%%%%%%%%%%%%%%%%
%\clearpage
\section{Digital control} \label{sec:digital_control}

In engineering practice, the most common controller type is digital controller. 
They give several advantages over the traditional analog controllers, which they usually imitate. 
Multiple type controller can be implemented on the same microcontroller, it can be programmed to act as a P, PD, PID or any more advanced controller. 
It is very easy to reuse a digital controller for a different purpose. Moreover, the control parameters and the reference value can be dynamically set. 
These really ease to tune them properly and precisely; or implement more sophisticated control laws like adaptive control.

However there are some digital effects, which are not in the focus of attention usually, but they can affect the system performance. 
These digital effects are the discretization of measurements and a discrete time delay.
There are practical examples, when the otherwise stable system actuated by digital controller looses stability just because of these digital effects \cite{stepan1989retarded,stepan2001vibrations}; but the reverse case can happen also, it is possible that an unstable system gains stability due to digital effects \cite{veraszto2017nonlinear}. 
So digital effects can be both beneficial and disadvantageous for different cases.

The discretization is always present at two different levels.
One is the resolution of measurement instruments, other is the microcontroller's finite binary representation of float numbers.
The later effects both the measurement and the calculation of the control law causing numerical errors.
These problems can be overcome by using a better equipment, but the problem remains in some hopefully negligible extent.

The discrete time delay is caused by the periodic sampling. 
It takes time to the microcontroller to do the measurement, do the analog digital conversion, calculate the control signal and give this signal to the actuator. This cycle then starts again while the control signal remains constant, this results in a piecewise constant excitation of the system.

We deal only with the sampling delay during our analysis, we assume a constant sampling frequency $f_s$, the  sampling time is 
\begin{align}
	\tau = \dfrac{1}{f_s}\,.
\end{align}

%\noindent for which the average time delay of the system is $1.5\tau$ as stated above.
\noindent The time at the $j^\mathrm{th}$ sampling is \begin{align}
	t_j = j\tau\,. 
\end{align}

\noindent
The actuating torque $M$ of electric motor considering the back-EMF effect and digital controller is  
\begin{align}
	M(t) = N\Uin(t_{j-1})-K\dot{\varphi}(t)\,,\quad t\in[t_{j},t_{j+1})\,;
\end{align}

\noindent similarly to equation (\ref{eq:electic_motor}), but here the input voltage is piecewise constant with one sampling time delay, while the back-EMF related term remains analog.
We chose the \PDD{} control law, because for $D_2=0$, it gives back the simple PD controller, so we can check both of them together.
The input voltage corresponding to the chosen \PDD{} control law is
\begin{align}
	\Uin(t_{j-1}) = -P_1\theta(t_{j-1})-D_1\dot \theta(t_{j-1}) + D_2 \dot \varphi (t_{j-1}) =: u_j\,, \quad t\in[t_{j},t_{j+1})\,, 
	\label{eq:digital_conrol_law_PDD} 
\end{align}

\noindent this can be substituted back to obtain the control torque
\begin{align}
	M(t) = Nu_j -K\dot{\varphi}(t)\,,\quad t\in[t_{j},t_{j+1})\,.
\end{align}

The linearized differential equations around the upward equilibrium point of Furuta pendulum omitting the damping terms ($b_1:=0, b_2:=0$) for simplicity is 
\begin{subequations}
	%\label{eq:eq_of_motion_both_linearized}
	\begin{align}
		&J_a \ddot\varphi 
		- m_2 r l  \ddot\theta  
		= Nu_j -K\dot{\varphi}\,,
		%
		%\label{eq:eq_of_motion_fi_linearized}
		\\[1ex]
		%
		& J_p \ddot\theta 
		- m_2rl \ddot \varphi
		- m_2gl \theta 
		= 0\,,\quad t\in[t_{j},t_{j+1})\,.
		%\label{eq:eq_of_motion_theta_linearized}
	\end{align}
\end{subequations}

\noindent
Using the Cauchy transformation this can be transformed to a first order linear differential equation system by introducing $\phi:= \dot \varphi$ and $\Theta := \dot \theta$. The resulting differential equation system in matrix form is
\begin{align}
	\begin{bmatrix}
		\dot \varphi 	\\
		\dot \theta		\\
		\dot \phi		\\
		\dot \Theta
	\end{bmatrix}
	= 
	\begin{bmatrix}
		0 & 0 & 1 & 0 \\
		0 & 0 & 0 & 1 \\
		0 & A_{12} & A_{22} & 0 \\
		0 & A_{13} & A_{23} & 0 
	\end{bmatrix}
	\begin{bmatrix}
		\varphi 	\\
		\theta		\\
		\phi		\\
		\Theta
	\end{bmatrix} + 
	\begin{bmatrix}
		0\\
		0\\
		f_2\\
		f_3
	\end{bmatrix}u_j\,,\quad t\in[t_{j},t_{j+1})\,.
\end{align}

\noindent
It can be seen that the first column of the coefficient matrix is zero, the system is independent of the coordinate $\varphi$, this is a cyclic coordinate. We do not use $\varphi$ in the digital control law (\ref{eq:digital_conrol_law_PDD}), so for simplicity we can omit this first equation for further calculation. 
This yields 
\begin{subequations}
	
\begin{align}
	\underbrace{\begin{bmatrix}
		\dot \theta		\\
		\dot \phi		\\
		\dot \Theta
	\end{bmatrix}}_{\displaystyle \mathbf{\dot x}}
	= 
	\underbrace{\begin{bmatrix}
		 0 & 0 & 1 \\
		 A_{12} & A_{22} & 0 \\
		 A_{13} & A_{23} & 0 \\
	\end{bmatrix}}_{\displaystyle \mathbf A}
	\underbrace{\begin{bmatrix}
		\theta		\\
		\phi		\\
		\Theta
	\end{bmatrix}}_{\displaystyle \mathbf x\vphantom{A}} + 
	\underbrace{\begin{bmatrix}
		0\\
		f_2\\
		f_3
	\end{bmatrix}}_{\displaystyle \mathbf f}
	u_j,\quad t\in[t_{j},t_{j+1})\,,
\end{align}

\noindent where
\begin{align}
	A_{12} &= \dfrac{g}{r\Omega}\,,
	\\
	A_{13} &= \dfrac{g J_a} {m_2r^2l\Omega}\,,
	\\
	A_{22} &= -\dfrac{K J_p} {m_2^2r^2l^2\Omega}\,,
	\\
	A_{23} &= -\dfrac{K} {m_2rl\Omega}\,,
	\\
	f_2 &= \dfrac{J_p}{m_2rl\Omega}\,,
	\\
	f_3 &=\dfrac{1}{\Omega}\,,
	\\
	\Omega &= \dfrac{J_aJ_p}{m_2^2r^2l^2}-1\,.
\end{align}
\end{subequations}

\noindent The compact form of this is
\begin{align}
	\mathbf{\dot x}=\mathbf A\mathbf x + \mathbf f u_j,\quad t\in[t_{j},t_{j+1})\,.
\end{align}

\noindent The general solution can be composed by homogeneous and inhomogeneous (particular) parts

\begin{align}
	\mathbf x(t) = \mathbf x_\mathrm{H}(t) + \mathbf{x}_\mathrm{P}(t),\quad t\in[t_{j},t_{j+1})\,.
\end{align}


\noindent
The particular solution is constant, since it is inherited from the excitation
\begin{align}
	\mathbf{x}_\mathrm{P}(t)\equiv \mathbf{x}_\mathrm{P} = -\mathbf A^{-1} \mathbf f u_j,\quad t\in[t_{j},t_{j+1})\,.
\end{align}

\noindent
The homogeneous solution is the matrix exponential
\begin{align}
	\mathbf x_\mathrm{H}(t)= \mathbf C e^{\displaystyle \mathbf A t} = C_1e^{\displaystyle \lambda_1 t}\mathbf v_1 + C_2 e^{\displaystyle\lambda_2 t}\mathbf v_2 + C_3 e^{\displaystyle\lambda_3 t}\mathbf v_3,\quad t\in[t_{j},t_{j+1})\,,
\end{align}

\noindent where $\lambda_i$ are the eigenvalues, $\mathbf v_i$ are the corresponding eigenvectors of $\mathbf A$ matrix, $c_i$ come from the initial conditions.
It turned out that despite the similarities of elements of $\mathbf A$, this eigenvalues $\lambda_i$ are extremely complicated, these are just the general solution of the third order characteristic polynomial. There is no practical reason to express such a complicated terms in closed form, because later we cannot say any useful conclusion based on them. 

However, the eigenvalues and eigenvectors can be easily expressed numerically by substituting the mechanical parameters of an existing pendulum and using pure numerical methods, since matrix $\mathbf A$ contains only system parameters, but not any controller related terms. This is the reason why we skipped introducing the dimensionless time, it would appear in the eigenvalues and eigenvectors excluding the numerical evaluation.
All the numerical eigenvalue, eigenvector calculations were done by \texttt{numpy}'s built-in \texttt{linalg.eig} method, the substituted system parameters can be found in Table \ref{tab:system_params}.

The total solution becomes
\begin{align}
		\mathbf x(t)
		= 
		C_1e^{\displaystyle \lambda_1 t}\mathbf v_1 
		+ C_2 e^{\displaystyle\lambda_2 t}\mathbf v_2 
		+ C_3 e^{\displaystyle\lambda_3 t}\mathbf v_3
		-\mathbf A^{-1} \mathbf f u_j,\quad t\in[t_{j},t_{j+1})\,,
\end{align}

\noindent where $\lambda_i$, $\mathbf v_i$ and $\mathbf A^{-1} \mathbf f$ are known numerically for a specific pendulum,  the $c_i$ coefficients can be expressed from arbitrary initial condition \begin{align}
	\mathbf x(0) = \mathbf x_0 = C_1 \mathbf v_1 
	+ C_2 \mathbf v_2 
	+ C_3 \mathbf v_3
	-\mathbf A^{-1} \mathbf f u_j\,,\quad t=0,\ j=0
\end{align}
\noindent as
\begin{align}
	C_i = c_i(\mathbf x_0, u_j;\lambda_k, \mathbf v_k, \mathbf A^{-1}\mathbf f).
\end{align}

\noindent This can be substituted into the solution for one sampling period only as
\begin{align}
	\mathbf x(t)
	= 
	c_1e^{\displaystyle \lambda_1 t}\mathbf v_1 
	+ c_2 e^{\displaystyle\lambda_2 t}\mathbf v_2 
	+ c_3 e^{\displaystyle\lambda_3 t}\mathbf v_3
	-\mathbf A^{-1} \mathbf f u_j,\quad t\in[0,\tau)\,,
\end{align}

\noindent for which the solution at the end of the sampling cycle is 
\begin{align}
	\mathbf x(\tau)
	= 
	c_1e^{\displaystyle \lambda_1 \tau}\mathbf v_1 
	+ c_2 e^{\displaystyle\lambda_2 \tau}\mathbf v_2 
	+ c_3 e^{\displaystyle\lambda_3 \tau}\mathbf v_3
	-\mathbf A^{-1} \mathbf f u_j\,.
\end{align}

\noindent This can be expressed in a form 
\begin{align}
		\begin{bmatrix}
			\theta(\tau)		\\
			\phi(\tau)		\\
			\Theta(\tau)
		\end{bmatrix}
		= 
		\begin{bmatrix}
			B_{ 1 1 } & B_{ 2 1 } & B_{ 3 1 } & B_{ 4 1 }\\
			B_{ 1 2 } & B_{ 2 2 } & B_{ 3 2 } & B_{ 4 2 }\\
			B_{ 1 3 } & B_{ 2 3 } & B_{ 3 3 } & B_{ 4 3 }
		\end{bmatrix}
		\begin{bmatrix}
			\theta(0)		\\
			\phi(0)		\\
			\Theta(0) \\
			u_0
		\end{bmatrix},
\end{align}

\noindent representing a linear mapping from the initial to final state. This can be extended with the control law (\ref{eq:digital_conrol_law_PDD}) as

\begin{align}
		\begin{bmatrix}
			\theta(\tau)		\\
			\phi(\tau)		\\
			\Theta(\tau)	\\
			u_1
		\end{bmatrix}
		= 
		\underbrace{\begin{bmatrix}
				B_{ 1 1 } & B_{ 2 1 } & B_{ 3 1 } & B_{ 4 1 }\\
				B_{ 1 2 } & B_{ 2 2 } & B_{ 3 2 } & B_{ 4 2 }\\
				B_{ 1 3 } & B_{ 2 3 } & B_{ 3 3 } & B_{ 4 3 }\\
				-P_1	  & D_2       & -D_1      & 0
		\end{bmatrix}}_{\displaystyle \mathbf B}
		\begin{bmatrix}
				\theta(0)		\\
				\phi(0)		\\
				\Theta(0) \\
				u_0
		\end{bmatrix},
\end{align}

\noindent where all $B_{ik}$ is dependent on the sampling time $\tau$ and $\mathbf A^{-1} \mathbf f$.
This is a discrete-time system and the $\mathbf B$ is called the quotient matrix, its eigenvalues $\mu_k$ are called characteristic multiplicators. This discrete system is stable if 
\begin{align}
	|\mu_k| < 1,\quad k=1,2,3,4\,.
	\label{eq:condition_muk}
\end{align}

For an analytic closed form solution these  conditions can be transformed by Möbius transformation in order to be able to check the stability by Routh-Hurwitz criterion.
%
But our result is mainly numerical and specific to a given pendulum, so these conditions were checked numerically by substituting lot of $P_1$, $D_1$, $D_2$ and $\tau$ values, calculating the $\mu_k$ characteristic multiplicators with numerical method for each case and checking the conditions (\ref{eq:condition_muk}). 

The critical sampling time is identified by choosing the maximal sampling time $\tau$ for which the conditions are satisfied.
The maximal satisfactory sampling time for our Furuta pendulum is
\begin{align}
	\tau_\mathrm{crit} = 0.142 \,\mathrm s\,, 
\end{align}

\noindent which is in the realistic order of magnitude. It is worth to note that the other parameters are appropriate in a very narrow range for this time delay.

A lower boundary for parameter $D_2$ were found around
\begin{align}
	D_{2,\min}\approx 0.01072\,\mathrm{Vs/rad}\,,
\end{align}
which means that this term is necessary again to stabilize the Furuta pendulum at the upward position considering the back-EMF effect of DC motor. The discrete time delay does not help this case.

We have three control parameters $P_1$, $D_1$, $D_2$ and the additional time delay $\tau$, it is very hard to show a comprehensive stability chart this case, but an example is presented in Figure \ref{fig:digit_cont_stab_chart}.
\begin{figure}[b!]
	\centering
	\includegraphics[width=0.96\linewidth]{figures/all_digital_stab_chart_BIG_outer_fix_g_with_units}
	\hspace{-1.8em}
	\llap{ \raisebox{1.5cm}{%  move next graphics to top right corner
			\includegraphics[width=0.835\linewidth]{all_digital_stab_chart_fix_g_with_units}%
		}}
	\caption{4D stability chart for the digital \PDD{} controller. The greenish part is stable, the reddish is unstable.}
	\label{fig:digit_cont_stab_chart}
\end{figure}


