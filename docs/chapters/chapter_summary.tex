\chapter{Summary}\label{chap:summary}

\vspace{2.5eM}

This work begins with the derivation of the equations of motion of Furuta pendulum. These equations capture the full dynamic behavior. Due to the complexity, we applied further well-established mechanical assumptions to get approximate equations, which are simpler. We also derived the linearized equations around the upward equilibrium.
\\
The Furuta pendulum is a cyclic system, we managed to eliminate the cyclic coordinate from the nonlinear equation of motion via the Routh reduction. 
This resulted in the equation of essential motion. 
We identified further equilibrium positions of the stationary motion and proved that the downward equilibrium position has a supercritical Pitchfork bifurcation at specific cyclic velocity. 
Above this critical speed the downward position becomes unstable.

We investigated the stabilization of upward position of Furuta pendulum. 
We considered the presence of back-EMF effect of a real electric motor. 
Both analog and digital controllers were investigated. For analog controllers we had to improve the PD control law, because this was unable to stabilize the pendulum with the presence of back-EMF effect. The introduced \PDD{} controller was appropriate and we obtained the stability chart.
\\
We considered the discrete time delay of digital controllers. 
The back-EMF effect led to additional problems.
We obtained stability chart via numerical method only for a specific pendulum with given system parameters. The critical time delay was identified and we found that it is necessary to use the \PDD{} control law in the digital controller also.

During experiments on the Furuta pendulum we had to face some difficulties. One component (namely the pendulum encoder) showed unreliable behavior, it had to be replaced. The replacement device worked as expected, so further experiments were possible. \\
We were able to develop embedded software using free software only. The implemented program is capable to tune the control parameters on the fly, the state variables and the controller performance can be monitored also.\\
The Coulomb friction and the gearbox backlash caused extra problems for the balancing. It was hard to tune the control parameters properly. 
\\
The stationary 
motions
of the Furuta pendulum were compared to the theoretical behavior;
the bifurcation diagram was reproduced quantitatively by experiments, and the effect of
dry-friction was qualitatively 
detected and identified. 
It was nice to see the bifurcation behavior in our
own experiments.
