
%%%%%%%%%%%%%%%%%%%%%%%%%%%%%%%%%%%%%%%%%%%%%%%%%%%%%%%%%%%%%%%%%%%%%%
%%%%%%%%%%%%%%%%%%%%%%%%%%%%%%%%%%%%%%%%%%%%%%%%%%%%%%%%%%%%%%%%%%%%%%
%%%%%%%%%%%%%%%%%%%%%%%%%%%%%%%%%%%%%%%%%%%%%%%%%%%%%%%%%%%%%%%%%%%%%%
%% ENGLISH ABSTACT							%
%\selectlanguage{english}					%
\chapter*{Abstract}							%
\addcontentsline{toc}{chapter}{Abstract}	%
%%%%%%%%%%%%%%%%%%%%%%%%%%%%%%%%%%%%%%%%%%%%%


	
	The Furuta pendulum is a two degree of freedom mechanical device, which consists of two rods, one of which rotates in the horizontal plane, and the other rod is placed at the end of the first rod and it rotates in the vertical plane. The actuator is usually an electrical DC motor, it rotates the first rod in the horizontal plane, the other rod rotates freely in the vertical plane. This type of systems is called underactuated. This mechanical system has strongly nonlinear dynamics due to the gravitational force and the special coupling between the degrees of freedom. The device is named after the Japanese professor Katsuhisa Furuta; he and his colleges used it first for benchmarking control strategies. Since then lot of papers have appeared about the control aspects of the Furuta pendulum (stabilization, swing-up). The dynamical model of the Furuta pendulum and the possible simplifications are not settled in the specialist literature. The studies are often contradictory, the results are different, the simplifications are neither well-established nor explained and/or justified properly.

	In this work, we derived the equations of motion by using the Lagrangian equation of the 2nd kind. The possible reduction of the equations was analyzed via the Routh method for the elimination of the so-called cyclic coordinate. The derived equations were modified to include the governing equation of the electrical actuator. The stabilization around the unstable equilibrium point was investigated and we obtained the stability regions for some analog control strategies (PD, PID, \PDD{}). In the case of real digital controllers, errors related to discretization and discrete time delay (sampling) are present. The equations were modified to to model these effects, too. We derived the stability chart of the delayed system and the critical time delay of the controller.
	
	
	
	
	
%%%%%%%%%%%%%%%%%%%%%%%%%%%%%%%%%%%%%%%%%%%%%%%%%%%%%%%%%%%%%%%%%%%%%%
%%%%%%%%%%%%%%%%%%%%%%%%%%%%%%%%%%%%%%%%%%%%%%%%%%%%%%%%%%%%%%%%%%%%%%
%%%%%%%%%%%%%%%%%%%%%%%%%%%%%%%%%%%%%%%%%%%%%%%%%%%%%%%%%%%%%%%%%%%%%%
%% MAGYAR ABSTACT							%
\begin{otherlanguage}{hungarian}			%
\chapter*{Kivonat}							%
\addcontentsline{toc}{chapter}{Kivonat}		%
%%%%%%%%%%%%%%%%%%%%%%%%%%%%%%%%%%%%%%%%%%%%%
%\cleardoublepage
%\thispagestyle{empty/plain/headings/myheadings}
%\thispagestyle{plain}
	
	A Furuta inga egy két szabadsági fokkal rendelkező mechanikai szerkezet, egy vízszintes síkban forgó rúd végére erősített függőleges síkban forgó rúdból áll. Az aktuátor szerepét rendszerint elektromos DC motor látja el, mely a vízszintes síkban forgó rudat mozgatja, a másik rúd szabadon forog a függőleges síkban. Az ilyen rendszereket alulaktuált rendszereknek nevezzük. Ez mechanikai rendszer erőteljes nemlineáris dinamikával rendelkezik, mely részben a gravitációs erőből részben pedig a szabadsági fokok közötti szokatlan dinamikai csatolásból ered. Az elnevezés Katsuhisa Furuta japán professzorhoz köthető, aki kollégáival 1992-ben először használta különböző szabályzási stratégiák tesztelésére, és azóta is számos tanulmány foglalkozott a Furuta inga szabályozástechnikai vonatkozásaival (stabilizálás, fellendítés). A szerkezet pontos dinamikai modellje, ennek lehetséges egyszerűsítései és dekompozíciója a mai napig nem egységesek a szakirodalomban. A témában megjelent tanulmányok és cikkek gyakran eltérő és egymásnak ellent mondó eredményennyel zárulnak, kétes és hiányosan indokolt egyszerűsítéseket tartalmaznak.
	
	Jelen dolgozatban a másodfajú Lagrange egyenlet segítségével levezetjük a dinamikai modellt. Ennek redukálási lehetőségeit Routh módszer segítségével elemeztük. A mechanikai egyenleteket az aktuátorból (DC motor) származó elektromos egyenlettel bővítettük. Az így kapott rendszer instabil egyensúlyi helyzete körüli szabályozására javaslatokat tettünk, megállapítottuk a különböző analóg szabályozási stratégiák (PD, PID, \PDD{}) stabilitási tartományit.
	
	A valós digitális szabályozók esetén diszkretizációból és a számítógépes feldolgozás késéséből adódó hibák lépnek fel. A digitális időkését figyelembe véve a szabályozást tovább bővítettük ennek megfelelően. Fél analitikus módszerrel a digitális időkéső szabályozáshoz is stabilitási térképeket számítottunk és megállapítottuk a digitális szabályozás kritikus időkésését is.
	
%%%%%%%%%%%%%%%%%%%%%%%%%%%%%%%%%%%%%%%%%%%%%%%%%%%%%%%%%%%%%%%%%%%%%%
%%%%%%%%%%%%%%%%%%%%%%%%%%%%%%%%%%%%%%%%%%%%%%%%%%%%%%%%%%%%%%%%%%%%%%
%%%%%%%%%%%%%%%%%%%%%%%%%%%%%%%%%%%%%%%%%%%%%%%%%%%%%%%%%%%%%%%%%%%%%%
%%  language back to english
\end{otherlanguage} % {magyar}
