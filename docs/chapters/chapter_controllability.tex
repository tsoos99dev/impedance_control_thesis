\chapter{Irányíthatóság}\label{chap:controllability}
A szabályozó akkor tudja követni a számára előírt impedancia modellt, 
ha megfelelő bemeneti feszültség alkalmazásával eljutttatható az előírt állapotba. 
A rendszer pólusai áthelyezhetők kell legyenek az impedancia modell pólusaiba. Ehhez 
a rendszer teljesen állapot irányítható kell legyen, azonban ez csak egy szükséges, 
de nem elégséges feltétele az impedancia modell alkalmazhatóságának.
A~\eqref{eq:state_space_generic} egyenlet állapottér modellje alapján a teljes állapot irányíthatóság feltétele, hogy
\begin{align}\label{eq:observability_generic}
    \left[\begin{array}{c|c|c}
        \BF{B} & \BF{AB} & \BF A^2 \BF B
    \end{array}\right]
\end{align}
legyen maximális rangú. 
Felhasználva~\eqref{eq:state_space} paramétereit a~\eqref{eq:observability_generic}-es feltételben szereplő mátrix
\begin{align}
    \begin{bmatrix}
        0 & 0 & \frac{K_\RM m}{JL} \\
        0 & \frac{K_\RM m}{JL} & -\frac{K_\RM m\left(B_\RM m L + JR\right)}{J^2 L^2} \\
        \frac{1}{L} & -\frac{R}{L^2} & -\frac{K^2_\RM m L + JR^2}{J L^3} \\
    \end{bmatrix},
\end{align}
alakba írható át. Továbbá ez a mátrix redukált lépcsős alakban
\begin{align}
    \begin{bmatrix}
        1 & 0 & 0 \\
        0 & 1 & 0 \\
        0 & 0 & 1 \\
    \end{bmatrix},
\end{align}
mely mátrix rangja megegyezik sorainak számával, így az irányíthatósági feltétele teljesül.