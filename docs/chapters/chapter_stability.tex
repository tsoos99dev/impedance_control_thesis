\chapter{Stabilitásvizsgálat időkéséssel}\label{chap:time_delay_stability}

A modell és a motor paramétereken kívül a rendszer különböző elemeiben megjelenő időkésés 
is hatással van a stabilitásra. Elegendően nagy időkésés mellett nem csak a mozgásra előírt 
feltételek sérülnek, hanem teljesen instabillá válhat a rendszer. Az időkésés hatását vizsgálja
a következő fejezet.

% \section{Vizsgálati módszerek összehasonlítása}
% A rúd differenciálegyenlete
% \begin{align}
%     \ddot\phi\left(t\right) - \frac{6g}{l}\phi\left(t\right) + 
%     \frac{6D}{ml}\dot\phi\left(t-\tau\right) + \frac{6P}{ml}\phi\left(t-\tau\right) = 0
% \end{align}
% A differenciálegyenlet Laplace transzformáltja
% \begin{align}
%     s^2\phi\left(s\right) - s\phi_0 - \dot\phi_0 - 
%     \frac{6g}{l}\phi\left(s\right) + 
%     \frac{6D}{ml}\left(se^{-s\tau}\phi\left(s\right)-\phi_{-\tau}\right) + 
%     \frac{6P}{ml}e^{-s\tau}\phi\left(s\right) = 0 
% \end{align}
% Kifejezve $\phi\left(s\right)$-t
% \begin{align}
%     \phi\left(s\right) = \frac{s\phi_0+\dot\phi_0+\frac{6D}{ml}\phi_{-\tau}}{s^2+\frac{6D}{ml}se^{-s\tau}+\frac{6P}{ml}e^{-s\tau}-\frac{6g}{l}}
% \end{align}
% A végérték frekvenciatartománybeli reprezentációban
% \begin{align}
%     \lim_{t \to \infty}\phi\left(t\right) = 
%     \lim_{s \to 0} \phi\left(s\right) = 
%     \frac{\dot\phi_0+\frac{6D}{ml}\phi_{-\tau}}{\frac{6P}{ml}-\frac{6g}{l}}
% \end{align}
% Az időkésést Taylor-sorral közelítve
% \begin{align}
%     \phi\left(t-\tau\right) = \phi\left(t-\right) - 
%     \frac{1}{1!}\dot\phi\left(t\right)\tau + 
%     \frac{1}{2!}\ddot\phi\left(t\right)\tau^2 - 
%     \frac{1}{3!}\dddot\phi\left(t\right)\tau^3 + \ldots
% \end{align}
% különböző rendű közelítéssekkel

\section{Stabilitás folytonos időben}

Időkésés megjelenhet a rendszer különböző pontjain. A motor kimenetének mérése, az adatáramlás 
a szabályozó részegységei között és a szabályozó jel kiszámítása mind időt vesznek igénybe. 
Ezeknek a hatásoknak az összeségét egy konstans átlag időkésés reprezentálja. A továbbiakban ez a 
paraméter legyen \(\tau_\RM d\).
\begin{figure}[ht]
    \begin{center}
    \includegraphics[width=12cm]{images/time_delay_example.png}
    \caption{Reprezentatív ábra az időkésés hatásáról folytonos időben}\label{fig:time_delay_example}
    \end{center}
\end{figure}
A szabályozó jel \(\tau_\RM d\) időegységgel eltolódva jelenik meg a motor bemenetén, ahogy a
~\ref{fig:time_delay_example}. ábra szemlélteti.

A stabilitásvizsgálat az időkéséssel kiegészített szögelfordulás-referencia jel átviteli függvényből 
kiindulva végezhető el. Az időkéséssel kiegészített blokk diagram egyszerűsített alakban 
a~\ref{fig:block_diagram_time_delay}. ábrán látható.
\begin{figure}[ht]
    \begin{center}
    \includegraphics[width=\textwidth]{images/block_diagram_time_delay.pdf}
    \caption{Az előírható tehetetlenség és a független pólus közötti összefüggés}\label{fig:block_diagram_time_delay}
    \end{center}
\end{figure}
A motor átviteli függvényei a~\eqref{eq:transfer_function}. egyenletekben szerepelnek. A szabályozó dinamikáját 
az átláthatóság érdekében a következő egyenletek összegzik:
\begin{equation}
    \begin{split}
        \dot{\tilde{\BF \eta}} &= \hat{\BF A} \tilde{\BF \eta} + \hat{\BF B} y + \hat{\BF F}_\RM V V + \hat{\BF F}_\RM \tau \tau_\RM e\,, \\
        \tilde{\BF \eta} &= \tilde{\BF x}_\RM b - \BF K_\RM e y\,, \\
        V &= K_\RM r \theta_\RM r - K_\RM c \tau_\RM e - \BF K \tilde{\BF x}\,,
    \end{split}
\end{equation}
ahol a megfigyelő belső dinamikáját leíró egyenletben megjelenő mátrixok~\eqref{eq:observer_params}-hez hasonlóan:
\begin{align}
    \begin{split}
        \hat{\BF A} &= \BF A_\RM{bb} - \BF K_\RM e \BF A_\RM{ab}\,,\\
        \hat{\BF B} &= \hat{\BF A} \BF K_\RM e + \BF A_\RM {ba} - \BF K_\RM e A_\RM {aa}\,,\\
        \hat{\BF F} &= \BF B_\RM b - \BF K_\RM e B_\RM a\,.
    \end{split}
\end{align}
A kimeneti feszültség egyenletében található becsült állapotvektor kifejezhető a megfigyelő belső állapotával és a mért kimenettel:
\begin{equation}
    \begin{split}
        V &= K_\RM r \theta_\RM r - K_\RM c \tau_\RM e - (K_a + \BF K_\RM b \BF K_\RM e) y - \BF K_\RM b \tilde{\BF \eta}\,,
    \end{split}
\end{equation}
behelyettesítve a megfigyelő dinamikai egyeneletébe:
\begin{align}\label{eq:observer_transfer_transformed}
    \begin{split}
        \dot{\tilde{\BF \eta}} &= \hat{\BF A} \tilde{\BF \eta} + 
        \hat{\BF B} y + 
        \hat{\BF F}_\RM V\left[K_\RM r \theta_\RM r - K_\RM c \tau_\RM e - (K_a + \BF K_\RM b \BF K_\RM e) y - \BF K_\RM b \tilde{\BF \eta}\right] + 
        \hat{\BF F}_\RM \tau \tau_\RM e \\
        &= (\hat{\BF A} - \hat{\BF F}_\RM V \BF K_\RM b)\tilde{\BF \eta} + 
        [\hat{\BF B} - \hat{\BF F}_\RM V (K_\RM a + \BF K_\RM b \BF K_\RM e)]y + 
        \hat{\BF F}_\RM V K_\RM r \theta_\RM r + 
        (\hat{\BF F}_\RM \tau - \hat{\BF F}_\RM V K_\RM c) \tau_\RM e\,.
    \end{split}
\end{align}
A szabályozó dinamikája ez alapján kifejezhető egy új állapottér modell formájában:
\begin{align}
    \begin{split}
        \dot{\tilde{\BF \eta}} &= \tilde{\BF A}\tilde{\BF \eta} + 
        \tilde{\BF B}_\RM y y + 
        \tilde{\BF B}_\RM r \theta_\RM r +
        \tilde{\BF B}_\RM \tau \tau\RM e\,, \\
        V &= \tilde{\BF C}\tilde{\BF \eta} + 
        \tilde{\BF D}_\RM y y + 
        \tilde{\BF D}_\RM r \theta_\RM r + 
        \tilde{\BF D}_\RM \tau \tau_\RM e\,,
    \end{split}
\end{align}
ahol a mátrix paraméterek~\eqref{eq:observer_transfer_transformed} szerint:
\begin{align}
    \begin{split}
        \tilde{\BF A} &= \hat{\BF A} - \hat{\BF F}_\RM V \BF K_\RM b\,,\\
        \tilde{\BF B}_\RM y &= \hat{\BF B} - \hat{\BF F}_\RM V (K_\RM a + \BF K_\RM b \BF K_\RM e)\,,\\
        \tilde{\BF B}_\RM r &= \hat{\BF F}_\RM V K_\RM r\,, \\
        \tilde{\BF B}_\RM \tau &= \hat{\BF F}_\RM \tau - \hat{\BF F}_\RM V K_\RM c\,, \\
        \tilde{\BF C} &= -\BF K_\RM b\,, \\
        \tilde{\BF D}_\RM y &= -(K_\RM a + \BF K_\RM b \BF K_\RM e)\,, \\
        \tilde{\BF D}_\RM r &= K_\RM r\,, \\
        \tilde{\BF D}_\RM \tau &= -K_\RM c\,.
    \end{split}
\end{align}
A szabályozó és a motor állapottér egyenletei időkéséssel kiegészítve a következő átviteli 
függvénnyel írhatók le:
\begin{align}
    \begin{split}
        y = \left(C^\RM V_\RM r \theta_\RM r + 
        C^\RM V_\RM \tau \tau_\RM e +
        C^\RM V_\RM y y\right)M^\RM y_\RM V e^{-s\tau_\RM d} +
        M^\RM y_\RM\tau \tau_\RM e\,,
    \end{split}
\end{align}
ahol a \(C^\RM n_\RM m\) és \(M^\RM n_\RM m\) a szabályozó és a motor átviteli függvényei adott 
bemenetekre és kimenetekre. Az átviteli egyenlet zárt körben:
\begin{align}
    \begin{split}
        y = \frac{1}{1 - M^\RM y_\RM V C^\RM V_\RM y e^{-s\tau_\RM d}}
        \left[
            M^\RM y_\RM V C^\RM V_\RM r \theta_\RM r e^{-s\tau_\RM d} + 
            \left(M^\RM y_\RM\tau + M^\RM y_\RM V C^\RM V_\RM \tau e^{-s\tau_\RM d}\right) \tau_\RM e
        \right]
    \end{split}
\end{align}
alakban írható le. A motor és a szabályozó átviteli függvényei bemenettől és kimenettől 
függetlenül ugyanazokkal a pólusokkal rendelkeznek, így mindkét bemenetre ugyanazok a stabilitási 
feltételek érvényesek. Az átviteli függvény pólusait a következő egyenlet megoldásai adják:
\begin{align}\label{eq:delay_characteristic}
    \begin{split}
        &s^5 + 
        \left(c_{41} + c_{42}\frac{B_\RM e}{M_\RM e}\right)s^4 +
        \left(c_{31} + c_{32}\frac{B_\RM e}{M_\RM e} + c_{33}\frac{K_\RM e}{M_\RM e}\right)s^3 + \\
        &\left[c_{21} + c_{22}\frac{B_\RM e}{M_\RM e} + c_{23}\frac{K_\RM e}{M_\RM e} + \left(
        c_{24} + c_{25}\frac{B_\RM e}{M_\RM e} + c_{26}\frac{K_\RM e}{M_\RM e}\right)e^{-s\tau_\RM d}\right]s^2 + \\
        &\left[c_{11} + c_{12}\frac{B_\RM e}{M_\RM e} + c_{13}\frac{K_\RM e}{M_\RM e} + \left(
        c_{14} + c_{15}\frac{B_\RM e}{M_\RM e} + c_{16}\frac{K_\RM e}{M_\RM e}\right)e^{-s\tau_\RM d}\right]s + 
        c_0 \frac{K_\RM e}{M_\RM e} = 0\,,
    \end{split}
\end{align}
ahol az együtthatókat a ... táblázat tartalmazza. D-szeparációt alkalmazva az egyenlet valós és 
képzetes részei:
\begin{align}\label{eq:delay_complex_separation}
    \begin{split}
        a_4\omega^4-(a_{21} + a_{22}\cos{\omega\tau_\RM d})\omega^2+a_{12}\omega\sin{\omega\tau_\RM d} + a_0\cos{\omega\tau_\RM d} &= 0 \\
        \omega^5 -a_3\omega^3 + a_{22}\omega^2\sin{\omega\tau_\RM d} + (a_{11} + a_{12}\cos{\omega\tau_\RM d})\omega - a_0\sin{\omega\tau_\RM d}  &= 0 \,,
    \end{split}
\end{align}
ami \(s=jw\) behelyettesítés után adódik. Az új együtthatók~\eqref{eq:delay_characteristic} alapján:
\begin{align}
    \begin{split}
        a_4 &= c_{41} + c_{42}\frac{B_\RM e}{M_\RM e}\\ 
        a_3 &= c_{31} + c_{32}\frac{B_\RM e}{M_\RM e} + c_{33}\frac{K_\RM e}{M_\RM e}\\ 
        a_{21} &= c_{21} + c_{22}\frac{B_\RM e}{M_\RM e} + c_{23}\frac{K_\RM e}{M_\RM e}\\ 
        a_{22} &= c_{24} + c_{25}\frac{B_\RM e}{M_\RM e} + c_{26}\frac{K_\RM e}{M_\RM e}\\ 
        a_{11} &= c_{11} + c_{12}\frac{B_\RM e}{M_\RM e} + c_{13}\frac{K_\RM e}{M_\RM e}\\ 
        a_{12} &= c_{14} + c_{15}\frac{B_\RM e}{M_\RM e} + c_{16}\frac{K_\RM e}{M_\RM e}\\ 
        a_0 &= c_0 \frac{K_\RM e}{M_\RM e}\,.
    \end{split}
\end{align}
A~\eqref{eq:delay_complex_separation}-ben szereplő egyenletrendszert megoldva \(\frac{B_\RM e}{M_\RM e}\)
és \(\frac{K_\RM e}{M_\RM e}\) kifejezésekre egy paraméteres görbe adódik. 
A független paraméter \(\omega\) a stabilitás határán fellépő csillapítatlan rezgés frekvenciájval
arányos. A görbe minden pontjához egy tisztán képzetes póluspár vagy nulla kapcsolódik. 
A nulla megoldások a \(\frac{K_\RM e}{M_\RM e} = 0\) egyenesen helyezkednek el. 
A kapott görbét a~\ref{tab:delay_stab_params}. táblázat paramétereit behelyettesítve 
a~\ref{fig:time_delay_stab_map}. ábra mutatja. A kontúrvonalakon szereplő értékek \(\tau_\RM d\)
időkésés különböző értékeit jelölik szekundumban kifejezve.

\begin{table}[ht]
    \small\centering
    \caption{A folytonos idejű stabilitásvizsgálatnál alkalmazott paraméterek}\label{tab:delay_stab_params}
    \tabcolsep=1pt
    \begin{tabular}{l>{~}l>{\quad}rl}
        \toprule
        \multicolumn{2}{c}{Szimbólum és paraméter név} & \multicolumn{2}{c}{Érték} \\ \midrule
        \(J\) & Motor tehetetlensége & 0.01 & \(\RM{kg\cdot m^2}\) \\
        \(K_\RM m\) & Motor nyomatékállandója & 0.01 & \(\RM{Nm\cdot A^{-1}}\) \\
        \(B_\RM m\) & Motor modell viszkózus csillapítása & 0.1 & \(\RM{kg\cdot m^2\cdot s^{-1}}\) \\
        \(L\) & Motor induktivitása & 0.2 & H \\
        \(R\) & Motor ellenállása & 1 & \(\Omega\) \\
        \(p\) & További pólusok & -15 & \(\RM{rad \cdot s^{-1}}\) \\
        \(K_\RM c\) & Nyomaték kompenzációs együttható & -50 & \(\RM{V \cdot N^{-1} m^{-1}}\) \\
        \bottomrule
    \end{tabular}
\end{table}

\begin{figure}[ht]
    \begin{center}
    \includegraphics[width=\textwidth]{images/time_delay_stab_map.png}
    \caption{}\label{fig:time_delay_stab_map}
    \end{center}
\end{figure}

Időkésés nélkül a görbe pontjai a \(\frac{B_\RM e}{M_\RM e} = 0\) egyenesen maradnak.
A behatárolt stabilitástartomány éppen megegyezik a Routh--Hurwitz kritérium által 
kapott tartománnyal, tehát:
\begin{equation}
    M_\RM e > 0,~~B_\RM e > 0,~~K_\RM e > 0.
\end{equation}
Az időkésés növekedésével egyre szűkül a stabil tartomány, egy kritikus érték alatt azonban
csak \(\frac{K_\RM e}{M_\RM e}\) kifejezésre adódik maximum, \(\frac{B_\RM e}{M_\RM e}\)
tetszőleges pozítív értéket felvehet. A kritikus időkésés felett egy véges területet határolnak be 
a görbe \(\omega > 0\) és \(\omega = 0\) szegmensei. 

\section{Stabilitás diszkrét időben}