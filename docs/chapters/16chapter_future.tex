\chapter{Jövőbeli munka}\label{chap:conclusion}

A kapott eredményeket jelen állapotában nem lehet közvetlenül alkalmazni a REHAROB kézmodulján. 
Ehhez a következő lépés, hogy  általánosítsuk a modellt és az eredményeket több szabadsági fokú rendszerekre, ami már megfeleltethető a valódi kézmodulnak. 

A kapott diszkrét idejű stabilitástérképet alaposabban be lehetne járni, 
a motorparaméterek pontosabb meghatározása után. Különböző időkésésekre össze lehetne hasonlítani 
a mért és a szimulált beállási időket. A stabilitás határán ki lehetne mérni a rezgési frekvenciát, 
illetve annak megváltozását a határon végighaladva. 

A beállási időn kívül más rendszerparamétereket 
is meg lehetne vizsgálni, mint a felfutási idő vagy a maximális túllövés. Kiterjeszthető a modell 
nemlineáris hatások bevonásával, ilyen lehetne a Coulomb-súrlódás vagy a szaturáció, esetleg a 
szabályozó jel diszkretizációja. 

A külső nyomaták kompenzációja ugyan bekerült a 
szabályozó modelljébe, de nem maradt idő a kompenzáció tesztelésére. Végül előfordulhat, 
hogy az impedancia modell paramétereit online kell változtatni, például ha ``puhább'' választ 
szeretnénk kapni egy felülettel való kontakt során. Ennek az implementációját és a stabilitásra 
gyakorolt hatását is külön meg kell vizsgálni.
